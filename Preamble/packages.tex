% ---most of the packages i'm using are copy-pasted from an existing template found in Overleaf--- %

\usepackage{newtx} % a fonts package

\usepackage{amsbsy} % for producing bold maths symbols
\usepackage{amsfonts} % an extended set of fonts for maths
\usepackage{amssymb} % various maths symbols
\usepackage{amsthm} % for producing theorem-like environments

\usepackage{array} % for nice array and tabular environments
\usepackage{tabularx} % for tables
\usepackage{float} % makes dealing with floats (e.g. tables and figures) easier
\usepackage{framed} % for producing framed boxes

\usepackage{datetime2} % managing dates and times
\usepackage{delimseasy} % makes easy the manual sizing of brackets, square brackets, and curly brackets
\usepackage{enumitem} % customing list environments

\usepackage{graphicx} % for including graphics in the document
\usepackage{hyperref} % automatically produce hyperlinks for cross-references
\usepackage{mathtools} % package for maths (fixes some deficiences of amsmath so is preferred)
\usepackage{cleveref} % makes cross-referencing easier (must be loaded after hyperref and mathtools)
\usepackage{microtype} % better font sizing (extremely helpful with long equations!)

\usepackage{pdfpages} % for including pdf documents inside the compiled pdf
\usepackage{pgf} % produce pdf graphics using LaTeX
\usepackage{pgfplots} % create normal/logarithmic plots in two and three dimensions
\pgfplotsset{compat=1.18} % sorts out the compatability warning
\usepackage{physics} % useful for vector calculus and linear algebra symbols

\usepackage{tikz-3dplot} % for producing 3d plots
\usepackage{tikz} % for drawing graphics in LaTeX
\usepackage{tkz-base} % drawing with a Cartesian coordinate system
\usepackage{tkz-euclide} % drawing in Euclidean geometry
\usetikzlibrary{shapes.geometric,fit}
\usepackage{cancel} % for arrows

\usepackage{xcolor, soul} % for highlights
\sethlcolor{yellow} % highlight color
\usepackage[theorems]{tcolorbox} % for producing coloured boxes
\tcbuselibrary{theorems} % theorems with tcolorbox
